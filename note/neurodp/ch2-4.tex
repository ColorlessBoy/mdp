% Neuro Dynamic Programming

\section{Introduction}%

\section{Dynamic Programming}%

\begin{definition}
    \textbf{(Proper stationary policy).} (Reach termination state 0 w.p.1)
    \[
       \rho^{\pi} = \max_{i = 1, \ldots, n} P^{\pi} \left\{ s_n \ne 0 | i_0 = i \right\} < 1
    \]
\end{definition}

In stochastic shortest path problems, we have two assumptions:
\begin{itemize}
    \item There exists at least one proper policy;
    \item For every improper policy $ \pi $, the corresponding cost-to-oo $ J^{\pi}(i) $ is infinite for at least one state i.
\end{itemize}

\textbf{Policy Iteration as an Actor-Critic System}

\begin{itemize}
    \item Critic: policy evaluation;
    \item actor: policy improvement.
\end{itemize}

\section{Neural Network Architectures and Training}%

Trivial. Using some function to approximate $ V^{\pi}, V^{*}, Q^{\pi}, Q^{*} $.
This book uses neural network.

\section{Stochastic Iterative Algorithms}%

Suppose that we are interested in solving a system of equations of the form
\[
    Hr = r,
\]
where $ H $ is a function from $ \mathbb{R}^{n} $ into itself.
If $ Hr = r - \nabla f(r) $, the solution of the system $ Hr = r $ is of the form
\[
    \nabla f(r) = 0,
\]
Then it's sometime minimize the cost function $ f $.

One possible algorithm for solving the system $ Hr = r $ is provided by the iteration
\[
    r_{t+1} = H r_{t}, \quad or \quad r_{t+1} = (1 - \gamma)r_{t} + \gamma H r_{t}.
\]
the second method reduces to the gradient method if $ Hr = r - \nabla f(r) $.

Sometimes an exact evaluation of $ Hr $ is difficult but that we have access to a random variable s of the form $ s = Hr + w $, where w is a random noise term.
Then we obtain stochastic iterative or stochastic approximation algorithm
\[
    r_{t+1} = (1 - \gamma)r + \gamma(Hr + w).
\]

A more concrete setting is obtained as follows. Let $ v $ be a random variable with a known probability distribution $ p(v|r) $ that depends on r. Suppose that we are interested in solving:
\[
    \mathbb{E}_{v \sim p(v|r) } \left[ g(r,v) \right] = r,
\]
where g is a known function.
We can use preceeding algorithm:
\[
    r_{t+1} = (1 - \gamma) r_{t} + \gamma \mathbb{E}_{v \sim p(v|r)} \left[ g(r,v) \right].
\]
We can estimate $ \mathbb{E}_{v \sim p(v|r)} \left[ g(r,v) \right] \approx \frac{1}{k}  \sum^{k}_{i=1} g(r, \tilde v_i)$.
We get Robbins-Monro stochastic approximation algorithm ($ k=1 $),
\[
    r_{t+1} = (1 - \gamma) r_{t} + \gamma g(r, \tilde v),
\]
which is a special case of the algorithm $ r_{t+1} = (1 - \gamma)r_{t} + \gamma \left( Hr_{t} + w \right) $,
where $ Hr = \mathbb{E}_{v\sim p(v|r)}\left[ g(r,v) \right] $, and $ w = g(r, \tilde v) - \mathbb{E}\left[ g(r,v) \right] $.

\subsection{THE BASIC MODEL}%

Let $ T^i $ be the set of times at which $ r(i) $ updates:
\[
    r_{t+1}(i) =
    \begin{cases}
        r_t(i), & t \notin T^i\\
        (1 - \gamma_t(i))r_t(i) + \gamma_t(i) \left( (Hr_t)(i) + w_t(i) \right),& t \in T^i
    \end{cases}
\]

Assumption: $ \sum^{\infty}_{t=0} \gamma_t(i) = \infty $ and $ \sum^{\infty}_{t=0} \gamma^2_t(i) < \infty $.

\subsection{CONVERGENCE BASED ON A SMOOTH POTHENTIAL FUNCTION}%

\[
    r_{t+1} = r_t + \gamma_t \delta_t, \quad \delta_t = H r_t - r_t + w_t.
\]
Let $ \mathcal{H}_t $ denote the history of the algorithm
\[
    \mathcal{H}_t = \left\{ r_0, \ldots, r_t, \delta_0, \ldots, \delta_{t-1}, \gamma_0, \ldots, \gamma_t \right\}.
\]

\begin{assumption}
    Exist function $ f: \mathbb{R}^{n} \mapsto \mathbb{R} $, with the following properties:
    \begin{enumerate}
        \item $ \forall \mathbb{R}^{n}, f(r) \ge 0 $;
        \item $ \Arrowvert \nabla f(r_1) - \nabla f(r_2) \Arrowvert \le L {\Arrowvert r_1 - r_2 \Arrowvert}_2 $;
        \item (Pseudogradient property) $ c {\Arrowvert \nabla f(r_t) \Arrowvert}^2_2 + \langle \nabla f(r_t), \mathbb{E}\left[ \delta_t | \mathcal{H}_t \right] \rangle \le 0$
        \item $ \mathbb{E}\left[ {\Arrowvert \delta_t \Arrowvert}_2^2 | \mathcal{H}_t \right] \le K_1 + K_2 {\Arrowvert \nabla f(r_t) \Arrowvert}^2_2 $
    \end{enumerate}
\end{assumption}

\begin{proposition}
    Consider the algorithm $ r_{t+1} = r_{t} + \gamma_t s_t $, if $ \sum^{\infty}_{t=0} \gamma_t = \infty $ and $ \sum^{\infty}_{t=0} \gamma^{2}_t < \infty $.
    Under preceeding assumption, the following hold with probability 1:
    \begin{itemize}
        \item The sequence $ f(r_t) $ converges;
        \item $ \lim_{t \to \infty} \nabla f(r_t) = 0 $;
        \item Every limit point of $ r_t $ is a stationary point of f.
    \end{itemize}
\end{proposition}

\begin{example}
    \textbf{(Stochastic Gradient Algorithm).}
    \[
       r_{t+1} = r_{t} + \gamma_t \delta_t, \quad \delta_t = - \left( \nabla f(r_t) + w_t \right)   
    \]
    Assumption:
    \begin{enumerate}
        \item $ \sum^{\infty}_{t=0} \gamma_t = \infty, \quad \sum^{\infty}_{t=0} \gamma^2_{t} < \infty $;
            \item f is nonnegative and has a Lipschitz continuous gradient;
            \item $ \mathbb{E}\left[ w_t | \mathcal{H}_t \right] = 0,\quad \mathbb{E}\left[ \Arrowvert w_t \Arrowvert^2 | \mathcal{H}_t \right] \le A + B {\Arrowvert \nabla f(r_t) \Arrowvert}^2_2 $;
    \end{enumerate}
    We proof Assumption 1 is satisfied.
    \[
        \langle \nabla f(r_t), \mathbb{E}\left[ \delta_t | \mathcal{H}_t \right] \rangle
        = \langle \nabla f(r_t), -\nabla f(x_t) - \mathbb{E}\left[ w_t | \mathcal{H}_t \right] \rangle
        = - {\Arrowvert \nabla f(r_t) \Arrowvert}_2^2
    \]
    \begin{align*}
        \mathbb{E}\left[ {\Arrowvert \delta_t \Arrowvert}_2^2 | \mathcal{H}_t \right]
        =& {\Arrowvert \nabla f(r_t) \Arrowvert}_2^2 + \mathbb{E}\left[ \Arrowvert w_t \Arrowvert_2^2 | \mathcal{H}_t \right] + \langle 2 \nabla f(r_t), \mathbb{E}\left[ w_t | \mathcal{H}_t \right] \rangle\\
        =& \Arrowvert \nabla f(r_t) \Arrowvert^2_2 + A + B \Arrowvert \nabla f(r_t) \Arrowvert^2_2\\
        =& A + (B+1) \Arrowvert \nabla f(r_t) \Arrowvert^2_2
    \end{align*}
\end{example}

\begin{example}
    \textbf{(Estimate of an Unknown Mean).}
    For random variables $ v $ with unknow mean $ \mu $ and unit variance.
    \[
        r_{t+1} = (1 - \gamma_t) r_t + \gamma_t v_t.
    \]
    with assumption
    \begin{enumerate}
        \item $ \sum^{\infty}_{t=0} \gamma_t = \infty $ and $ \sum^{\infty}_{t=0} \gamma^2_t < \infty $;
    \end{enumerate}
    \begin{proof}
        \[
            r_{t+1} = r_t - \gamma_t (r_t - \mu) + \gamma_t(v_t - \mu)
        \]
        where $ f(r) = {(r - \mu)}^{2}/2  $, $ \nabla f(r_t) = (r_t - \mu) $.
        (The other assumptions in stochastic gradient algorithm are sastified naturally.)
    \end{proof}
\end{example}

\begin{example}
    \textbf{(Euclidean Norm Pseudo-Contractions).}
    \[
        r_{t+1} = (1 - \gamma_t) r_t + \gamma_t (H r_t + w_t),
    \]
    Assuming:
    \begin{enumerate}
        \item $ {\Arrowvert Hr - r^* \Arrowvert}_2 \le \beta {\Arrowvert r - r^* \Arrowvert}_2, \forall r \in \mathbb{R}^{n} $, $ 0 \le \beta < 1 $;
        \item $ \mathbb{E}\left[ w_t | \mathcal{H}_t \right] = 0 $;
            \item $ \mathbb{E} \left[ {\Arrowvert w_t \Arrowvert}^2_2 | \mathcal{H}_t  \right] \le A + B {\Arrowvert r_t - r^* \Arrowvert}^2_2 $
    \end{enumerate}

    The potential function is $ f(r) = \frac{1}{2} \Arrowvert r - r^* \Arrowvert^2_2. $, $ \delta_t = - r_t + H r_t + w_t $, then $ \mathbb{E} \left[ \delta_t | \mathcal{H}_t \right] = H r_t - r_t $.
    \begin{align*}
        \langle Hr - r^*, r - r^* \rangle \le& {\Arrowvert Hr - r^* \Arrowvert}_2 \Arrowvert r - r^* \Arrowvert_2 \le \beta {\Arrowvert r - r^* \Arrowvert}^2_2\\
        \langle Hr - r, r - r^* \rangle \le& -(1 - \beta) \Arrowvert r - r^* \Arrowvert^2_2\\
        \langle \mathbb{E} \left[ \delta_t | \mathcal{H}_t \right], \nabla f(r_t) \rangle \le& -(1-\beta) \Arrowvert \nabla f(r_t) \Arrowvert^2_2
    \end{align*}
    \[
    \mathbb{E}\left[ \delta^2_t | \mathcal{H}_t \right] = \mathbb{E}\left[ {(-r_t + H r_t)}^2 | \mathcal{H}_t \right] + \mathbb{E}\left[ \Arrowvert w_t \Arrowvert^2 | \mathcal{H}_t \right] \\
    \le {(H r_t - r_t)}^2 + A + B {\Arrowvert r_t - r^* \Arrowvert}_2^2
    \]
\end{example}

\subsubsection{Convergence Proofs}%
\label{ssub:convergence_proofs}

In this section, we discarded a suitable set of measure zero, and don't keep repeating the qualification `` with probability 1''.

\begin{theorem}
    \textbf{(Supermartingale Convergence Theorem).}
    Here is three sequences of random variables $ \left\{ X_t \right\}, \left\{ Y_t \right\} and \left\{ Z_t \right\} $.
    And let $ \mathcal{F}_t $ be set of random variables and $ \mathcal{F}_t \subset \mathcal{F}_{t+1} $.
    Suppose that
    \begin{enumerate}
        \item $ X_t, Y_t, Z_t $ are nonegative, and are functions of the random variables in $ \mathcal{F}_t $;
        \item $ \forall t $, we have $ \mathbb{E} \left[ Y_{t+1} | \mathcal{F}_t \right] \le Y_t - X_t + Z_t $;
        \item $ \sum^{\infty}_{t=0} Z_t < \infty $.
    \end{enumerate}
    Then we have $ \sum^{\infty}_{t = 0} X_t < \infty $, and the sequence $ Y_t $ converges to a nonegative random variable $ Y $, w.p.1.
\end{theorem}

\begin{theorem}
    \textbf{(Martigale Convergence Theorem)}
    Let $ \{ X_t \} $ be a sequence of random variables and let $ \mathcal{F}_t $ be set of random variables such that $ \mathcal{F}_t \subset \mathcal{F}_{t+1} $.
    Suppose that:
    \begin{enumerate}
        \item The random variable $ X_t $ is a function of the random variable in $ \mathcal{F}_t $;
        \item $ \mathbb{E}\left[ X_{t+1} | \mathcal{F}_t \right] = X_t $,
        \item $ \exists M < \infty $ such that $ \mathbb{E}\left[ \left| X_t \right| \right] \le M $.
    \end{enumerate}
    Then, the sequence $ X_t $ converges to a random variable $ X $, w.p.1.

    Now we begin proof the preceeding section.
    \begin{proof}
        By assumption, we have $ {\Arrowvert \nabla f(r_1) - \nabla f(r_2) \Arrowvert}_2 \le L \Arrowvert r_1 - r_2 \Arrowvert $, we have
        \[
            f(r_{t+1}) \le f(r_t) + \gamma_t \langle \nabla f(r), \delta_t \rangle + \frac{L}{2} \gamma^2_t {\Arrowvert \delta_t \Arrowvert}_2^2
        \]
        \begin{align*}
            \mathbb{E} \left[ f(r_{t+1} | \mathcal{F}_t) \right]
            \le& f(r_t) + \gamma_t \langle \nabla f(r_t), \mathbb{E} \left[ \delta_t | \mathcal{F}_t \right] \rangle + \frac{L}{2} \gamma^2_t \left( K_1 + K_2 \Arrowvert \nabla f(r_t) \Arrowvert^2_2 \right)\\
            \le& f(r_t) - \gamma_t \left( c - \frac{L K_2 \gamma_t}{2}  \right) \Arrowvert \nabla f(r_t) \Arrowvert^2_2 + \frac{L K_1 \gamma^2_t}{2} \\
            =& f(r_t) - X_t + Z_t,
        \end{align*}
        where
        \[
            X_t =
            \begin{cases}
                \gamma_t \left( c - \frac{L K_2 \gamma_t}{2}  \right) \Arrowvert \nabla f(r_t) \Arrowvert^2_2, & if\ LK_2\gamma_t \le 2c, \\
                0, & otherwise.
            \end{cases}
        \]
        and
        \[
            Z_t = 
            \begin{cases}
                \frac{L K_1 \gamma^2_t}{2}, & if\ LK_2 \gamma_t \le 2c,\\
                \frac{L K_1 \gamma^2_t}{2} - \gamma_t \left( c - \frac{L K_2 \gamma_t}{2}  \right) \Arrowvert \nabla f(r_t) \Arrowvert^2_2, & otherwise
            \end{cases}
        \]
        Because $ \sum^{\infty}_{t=0} \gamma^2_t < \infty $, so after some finite time $ LK_2 \gamma_t \le 2c $, and $ Z_t = LK_1 \gamma^2_t / 2 $, and therefore $ \sum^{\infty}_{t=0} Z_t < \infty $.
        Thus, the supermartingale convergence theorem applies and shows that $ f(r_t) $ converges and $ \sum^{\infty}_{t=0} X_t < \infty $.

        Because $ X_t = \gamma_t \left( c - \frac{LK_2 \gamma_t}{2}  \right) {\Arrowvert \nabla f(r_t) \Arrowvert}_2^2 \ge \frac{c}{2} \gamma_t \Arrowvert \nabla f(r_t) \Arrowvert^2_2 $ after some finite time. Hence
        \[
            \sum^{\infty}_{t=0} \gamma_t \Arrowvert \nabla f(r_t) \Arrowvert^2_2 < \infty
        \]
        Because $ \sum^{\infty}_{t=0} \gamma_t = \infty $, $ \liminf_{t \rightarrow \infty} {\Arrowvert \nabla f(r_t) \Arrowvert}_2 = 0 $
        
        Let us denote $ \bar s_t = \mathbb{E}\left[ s_t | \mathcal{F}_t \right] $ and $ w_t = s_t - \bar s_t $, then
        \[
            \Arrowvert \bar s_t \Arrowvert^2_2 + \mathbb{E} \left[ {\Arrowvert w_t \Arrowvert}_2^2 | \mathcal{F}_t \right] = \mathbb{E} \left[ \Arrowvert s_t \Arrowvert^2_2 | \mathcal{F}_t \right] \le K_1 + K_2 {\Arrowvert \nabla f(r_t) \Arrowvert}_2^2
        \]
        
        We need take a break and proof another lemma
        \begin{lemma}
            $ u_t = \sum^{t-1}_{\tau = 0} \chi_\tau \gamma_\tau w_{\tau} $, converges w.p.1.
            where $ \chi_t = 1_{\left[ {\Arrowvert \nabla f(r_t) \Arrowvert}_2 \le \epsilon \right]} $.
            \begin{proof}
                We start the assumption $ \sum^{\infty}_{t=0} \gamma^2_t \le A < \infty $.
                \[
                    \mathbb{E}\left[ \chi_t \gamma_t w_t | \mathcal{F}_t \right] = \chi_t \gamma_t \mathbb{E}\left[ w_t | \mathcal{F}_t \right] = 0 \Rightarrow \mathbb{E}\left[ u_{t+1} | \mathcal{F}_t \right] = u_t
                \]
                If $ \chi_t = 0 $, then $ \mathbb{E}\left[ \Arrowvert u_{t+1} \Arrowvert^2_2 | \mathcal{F}_t \right] = \Arrowvert u_t \Arrowvert^2 $. If $ \chi_t = 1 $, we have
                \[
                    \mathbb{E}\left[ \Arrowvert u_{t+1} \Arrowvert^2_2 | \mathcal{F}_t \right]
                    = {\Arrowvert u_t \Arrowvert}_2^2 + \gamma^2_t \mathbb{E}\left[ {\Arrowvert w_t \Arrowvert}_2^2 | \mathcal{F}_t \right] \le {\Arrowvert u_t \Arrowvert}_2^2 + \gamma^2_t (K_1 + K_2 \epsilon^2)
                \]
                \[
                    \mathbb{E}\left[ {\Arrowvert u_t \Arrowvert}_2^2 \right] \le (K_1 + K_2 \epsilon^2) \mathbb{E} \left[ \sum^{t-1}_{\tau = 0} \gamma^2_\tau \right] \le (K_1 + K_2 \epsilon^2)A
                \]
                \[
                    \sup_t \mathbb{E}\left[ {\Arrowvert u_t \Arrowvert}^2 \right] \le \sup_t \mathbb{E} \left[ 1 + \Arrowvert u_t \Arrowvert^2_2 \right] < \infty
                \]
                Then we can use Martigale convergence theorem to $ u_t $ and get that $ u_t $ converges, w.p.1.
                
                We can assume that $ \sum^{t-1}_{\tau = 0} \gamma^2_\tau \le A < \infty $ and get the same result.
            \end{proof}
        \end{lemma}
        \textbf{I give up today.}.
    \end{proof}
\end{theorem}


