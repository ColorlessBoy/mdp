% Chapter5 Sinulation Methods for a Lookup Table Representation, Neuro Dynamic Programming.

\section{Sinulation Methods for a Lookup Table Representation}%


\subsection{SOME ASPECTS OF MONTE CARLO SIMULATION}%

Suppose that v is a random variable with an unknown mean $ m $ that we wish to estimate.
Monte-Carlo simulation is to generate a number of samples $ \left\{ v_1, \ldots, v_N \right\} $,
and estimate the mean of v by forming the sample mean
\[
    M_N = \frac{1}{N} \sum^{N}_{k=1} v_k = M_{N-1} + \frac{1}{N} (v_{N} - M_N).
\]

\textbf{(The Case of i.i.d. Samples)}:
\begin{itemize}
    \item $ \mathbb{E}\left[ M_N \right] = \frac{1}{N}  \sum^{N}_{k=1} \mathbb{E}\left[ v_k \right] = m $;
    \item $ Var(M_N) = \frac{1}{N^2} \sum^{N}_{k=1} Var(v_k) = \frac{\sigma^2}{N}  $.
\end{itemize}

\textbf{(The Case of Dependent Samples)}:
\begin{itemize}
    \item The estimator of mean remains unbiased;
    \item We can use a weighted average to lower the variance.
\end{itemize}

\textbf{(The Case of a Random Sample Size)}:

In this case, the number of samples N is itself a random variable.

If N is correlated with $ \left\{ v_1, \ldots, v_N \right\} $. Then the sample mean is unbiased, and the variance is $ \sigma^2 \mathbb{E}\left[ 1/N \right] $.

If N is correlated with $ \left\{ v_1, \ldots, v_N \right\} $, the sample mean maybe biased.

\begin{theorem}
If $ \left\{ v_1, \ldots, v_N \right\} $ is {i.i.d.} with finite mean, N depends upon given sequence, and $ \mathbb{E}\left[ N \right] < \infty $.
\begin{align*}
    & \mathbb{E}\left[ \sum^{N}_{k=1} v_k \right] = \sum^{\infty}_{k=1} P(N \ge k) \mathbb{E} \left[ v_k | N \ge k \right]
    = \mathbb{E}\left[ v_1 \right] \sum^{\infty}_{k=1} P\left( N \ge k \right)\\
    =& \mathbb{E} [v_1] \sum^{\infty}_{k = 1} \sum^{\infty}_{n=k} P(N = n)
    = \mathbb{E}\left[ v_1 \right] \sum^{\infty}_{n=1} \sum^{n}_{k=1} P(N = n) = \mathbb{E}\left[ v_1 \right] \mathbb{E}\left[ N \right].
\end{align*}
\end{theorem}

\subsection{POLICY EVALUATION BY MONTE CARLO SIMULATION}%

\begin{itemize}
    \item State space: $ \left\{ 0, 1, \ldots, n \right\} $, where 0 is a cost-free absorbing state;
    \item The policy $ \pi $ is proper;
    \item For $ m $th trajectory is $ (s^m_0, s^m_1, \ldots, s^m_N) $;
    \item Cost of trajectory is $ c(s^m_0, m) = \sum^{N-1}_{t=1} g(s^m_t, s^m_{t+1}) $;
    \item Policy value: $ J^{\pi}(s) = \mathbb{E}^{\pi}\left[ c(s, m) \right] $.
\end{itemize}

\textbf{Algorithm1}:
\[
    \tilde J(i) = \frac{1}{K} \sum^{K}_{m=1} c(i,m)
\]
\[
    \tilde J^m(i) = \tilde J^{m-1}(i) + \frac{1}{m} (c(i,m) - \tilde J^{m-1}(i)), \quad s.t. J^{0}(i) = 0.
\]

\textbf{Algorithm2}:
Use the full trajectory.
\[
    J(i_k) = J(i_k) + \gamma(i_k) (g(i_k, i_k+1) + \cdots + g(i_{N-1}, i_N) - J(i_k))
\]

\textbf{Every-visit method} provides a biased estimator.

\textbf{Consistency of the Every-Visit Method}

Let $ c(s, k, m) $ mean, in $ m $th trajectory, the cost after visiting state $ s $ $ k $th times. And $ n^m_s $ means the total number of state s in $ m $th trajectory. Then every-visit method estimator is
\[
    \tilde J(s) = \frac{ \sum^{K}_{m=1} \sum^{n^m_s}_{k = 1} c(s, k, m) }{ \sum^{K}_{m=1} n^m_s} 
\]
When K is fixed, the extimator is biased. But if $ K \rightarrow \infty $, the estimator is unbiased:
\[
    \mathbb{E} \left[ \tilde J(s) \right] = \frac{\mathbb{E}\left[ \sum^{n^m_k}_{k = 1} c(s, k, m) | n_k \ge 1  \right]}{ \mathbb{E} \left[ n_k | n_k \ge 1 \right]} = \mathbb{E}\left[ c(s, 1, m) | n_k \ge 1 \right] = J^{\pi}(s)
\]

\textbf{The First-Visit Method}
\[
    \tilde J (s) = \frac{ \sum^{}_{m:n^m_s \ge 1} c(s, 1, m) }{ \sum^{K}_{m = 1} 1_{\left[ n^{m}_{s} \ge 1 \right]} } 
\]
which is unbiased.

The significance of the comparison of the every-visit and first-visit methods should not be overemphasized. For problems with large state space, the likelihood of a trajectory visiting the same state twice is usually quite small.

\subsubsection{Q-Factors and Policy Iteration}%

\[
    Q^{\pi}(s,a) = \sum^{}_{s' \in S} p_{s, s'}(a) (g(s, a, s') + J^{\pi}(s'))
\]

\subsection{TEMPORAL DIFFERENCE METHODS}%

{TD}: policy evaluation.

$ J^{m}_{k+1}(s^m_k) = J^{m}_{k}(s^m_k) + \gamma(g(s^m_k, s^m_{k+1}) + \cdots + g(s^{m}_{N-1}, s^{m}_{N}) - J^m_{k}(i_k)) $
\begin{align*}
    J^{m}_{k+1}(s^m_k) = J^{m}_{k}(s^m_k) + \gamma [ &
    (g(s^m_k, s^m_{k+1}) + J^m_{k+1}(s^m_{k+1}) - J^{m}_{k}(s^m_{k})) \\
    &+ (g(s^m_{k+1}, s^m_{k+2}) + J^m_{k+2}(s^m_{k+2}) - J^{m}_{k+1}(s^m_{k+1})) \\
    &+ \cdots \\
    &+ (g(s^m_{N-1}, s^m_{N}) + J^m_{N}(s^m_{N}) - J^{m}_{N-1}(s^m_{N-1}))
    ]
\end{align*}
where we have made use of $ J(s_N) = 0 $.
\[
    J^{m}_{k+1}(s^m_k) = J^{m}_{k}(s^m_k) + \gamma(d^{m}_{k} + d^{m}_{k+1} + \cdots + d^{m}_{N-1}),
\]
\[
    d^m_{k} = g(s^m_k, s^{m}_{k+1}) + J^{m}_{k+1}(s^m_{k+1}) - J^{m}_{k}(s^m_k)
\]

Make explanation complex, maybe approximation TD is more easy to understanding.

\section{Approximate DP with Cost-to-Go Function Approximation}%

This chapter is the core target I read this book.

The lookup table representation can be viewed as a limiting form of an appproximate representation.

\subsection{GENERRIC ISSUES-FROM PARAMETERS TO POLICIES}%

\subsection{APPROXIMATE POLICY ITERATION}%

\subsubsection{Approximate Policy Iteration Based on Monte Carlo Simulation}%







